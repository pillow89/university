Losen Sie das folgende lineare Gleichungssystem mit Hilfe des
Gauß-Algorithmus:
3x − 5y + z − 3t = 10
x + y + 2z + 4t = 4
x − y + z = 3
y + z + t = 5
Begr ̈unden Sie die einzelnen Schritte


[1]    3x − 5y + z − 3t = 10
[2]    x + y + 2z + 4t = 4     | + [3]
[3]    x − y + z = 3
[4]    y + z + t = 5
-------------------------
[1]    3x − 5y + z − 3t = 10
[2]    2x + 3z + 4t = 4
[3]    x − y + z + 0 = 3        | + [4]
[4]    y + z + t = 5
-------------------------
[1]    3x − 5y + z − 3t = 10
[2]    2x + 3z + 4t = 4
[3]    x + 2z + t = 3
[4]    y + z + t = 5           | * (5)
-------------------------
[1]    3x − 5y + z − 3t = 10
[2]    2x + 3z + 4t = 4
[3]    x + 2z + t = 3
[4]    5y + 5z + 5t = 25           | + [1]
-------------------------
[1]    3x − 5y + z − 3t = 10
[2]    2x + 3z + 4t = 4
[3]    x + 2z + t = 3              | *(-2)
[4]    3x + 6z + 2t = 35
-------------------------
[1]    3x − 5y + z − 3t = 10
[2]    2x + 3z + 4t = 4
[3]    -2x - 4z - 2t = -6           | + [2]
[4]    3x + 6z + 2t = 35
-------------------------
[1]    3x − 5y + z − 3t = 10
[2]    2x + 3z + 4t = 4
[3]    -z + 2t = -2
[4]    3x + 6z + 2t = 35       | * -(2/3)
-------------------------
[1]    3x − 5y + z − 3t = 10
[2]    2x + 3z + 4t = 4
[3]    -z + 2t = -2
[4]    -2x - 4z - 4/3t = 70/3     | + [2]
-------------------------
[1]    3x − 5y + z − 3t = 10
[2]    2x + 3z + 4t = 4
[3]    -z + 2t = -2            | *-(1) + [4]
[4]    - z + 8/3t = 82/3
--------------------------
[1]    3x − 5y + z − 3t = 10
[2]    2x + 3z + 4t = 4
[3]    2/3t = 88/3            | *(3/2)
[4]    - z + 8/3t = 82/3
--------------------------
[1]    3x − 5y + z − 3t = 10
[2]    2x + 3z + 4t = 4
[3]    t = 44
[4]    - z + 8/3t = 82/3      | t einsetzen (44)
-----------------------
[1]    3x − 5y + z − 3t = 10
[2]    2x + 3z + 4t = 4
[3]    t = 44
[4]    - z + 352/3 = 82/3     | - 352/3 | *(-1)
-----------------------
[1]    3x − 5y + z − 3t = 10
[2]    2x + 3z + 4t = 4
[3]    t = 44
[4]    - z + 352/3 = 82/3
----------------------
[1]    3x − 5y + z − 3t = 10
[2]    2x + 3z + 4t = 4      | t und z einsetzen
[3]    t = 44
[4]    z = 90
----------------------
[1]    3x − 5y + z − 3t = 10
[2]    2x + 270 + 176 = 4      | | -446 | / 2
[3]    t = 44
[4]    z = 90
----------------------
[1]    3x − 5y + z − 3t = 10    | x,y und t einsetzen
[2]    x = -221
[3]    t = 44
[4]    z = 90
----------------------
[1]    -663 − 5y + 90 − 132 = 10 | +705 | / (-5)
[2]    x = -221
[3]    t = 44
[4]    z = 90
----------------------
[1]    y = -143
[2]    x = -221
[3]    t = 44
[4]    z = 90

Aufgabe 1.2: Gegeben sei ein lineares Gleichungssystem mit m Gleichungen
und n Variablen. Sind die folgenden Aussagen wahr oder falsch?
Begrunden Sie Ihre Antworten.
1. Das Gleichungssystem ist nicht losbar, wenn m > n ist.
2. Das Gleichungssystem ist losbar, wenn n > m ist.
Ein Gleichungssystem in der Form von Definition 1.3 im Skript heißt
homogen, falls bi = 0 fur alle 1 ≤ i ≤ m gilt.
Wie verandern sich die Antworten, wenn das Gleichungssystem als homogen
vorausgesetzt wird?
Wie verandern sich die Antworten, wenn das Gleichungssystem als homogen
vorausgesetzt wird, und unter “losbar” die Existenz einer von der trivialen
Losung 0, . . . , 0 verschiedenen Losung verstanden wird?

Vermutung von mir:
1. Ich denke, dass es trotzdem lösbar ist, weil wenn ich nur 2 Variablen
in den Gleichungen hab. Aber zum Beispiel 3 Gleichungen, kann ich
irgendwann trotzdem einige Variablen rausbringen und in die anderen
Gleichungen ersetzen.
2. In der Regel nicht, weil die Variablen nicht, wie in dem
Gauß-Algorithmus rausgenommen werden können. Außer die Gleichungen sind
exakt wie in einer anderen Gleichung.

Bei homogen
1. Weiterhin lösbar, da egal was passiert alle Variablen 0 werden.
2. Ebenfalls wieder nicht lösbar, weil ich nicht vorraussagen kann, ob y
+ z = 0 dass beide 0 sind.

bei homogen + trivial
1. Weiterhin lösbar selbes wie bei homogen
2. Dieses Mal lösbar, weil die Möglichkeit besteht, dass die Variablen
dann 0 sind.



Ag: 1.3
Eigentlich nur, wenn die selben Vorgänge auf der anderen Ebene rückwärts
ebenfalls zum
Selben Ergebnis führt.




Ag: 1.4
[1] x −  y = 1 | *(-2)
[2] 2x − 2y = b

[1] -2x + 2y = -2
[2]  2x − 2y = b  | + [1]

[1] -2x + 2y = -2
[2]  0 = b - 2


a11x1 + a12x2 + : : : + a1nxn = b1
a21x1 + a22x2 + : : : + a2nxn = b2
:::
am1x1 + am2x2 + : : : + amnxn = bm.



also homogen wenn


x1 + x2 + x3 + x4 = 0
2x1 + 5x2 + 7x3 + 9x4 = 0
3x1 + 3x2 + 3x3 + 3x4 = 0



5x + 3y = 0
10x + 66y = 0


10x + 66y = 0
5x + 3y = 0





Aufgabe 1.5: Bestimmen Sie, f ̈ur welche t ∈ R das folgende lineare Glei-
chungssystem in Matrixdarstellung l ̈osbar ist, und geben Sie gegebenenfalls
die L ̈osung an.
[1] 2x1  4x2 2x3 = 12t
[2] 2x1 12x2 7x3 = 12t + 7 | *(-1) | + [1]
[3] 1x1 10x2 6x3 = 7t + 8
----------------------------------
[1] 2x1  4x2 2x3 = 12t
[2]    - 8x2 -5x3 = -7
[3] 1x1 10x2 6x3 = 7t + 8 | *(-2) | + [1]
----------------------------------
[1] 2x1  4x2 2x3 = 12t
[2]    - 8x2 -5x3 = -7        |
[3]    -16x2 -10x3 = -2t-16   | /(-2) | + [2]
-----------------------------------
[1] 2x1  4x2 2x3 = 12t
[2]    - 8x2 -5x3 = -7        |
[3]    0 = t+1   | /(-2) | + [2]


==========> t = -1?

Einsetzen

[1] 2x1  4x2 2x3 = -12
[2] 2x1 12x2 7x3 = -5  | *(-1) | + [1]
[3] 1x1 10x2 6x3 = 1
---------------------
[1] 2x1  4x2 2x3 = -12
[2]     -8x2 -5x3 = -7
[3] 1x1 10x2 6x3 = 1 | *(-2) | + [1]
---------------------
[1] 2x1  4x2 2x3 = -12
[2]     -8x2 -5x3 = -7
[3]   -16x2 -10x3 = -14 | /(-2) | + [2]
---------------------
[1] 2x1  4x2 2x3 = 12
[2]     -8x2 -5x3 = -7
[3]      8x2 +5x3 = 7
---------------------





