\title Informatik I: Einführung in die Programmierung:
2. Erste Schritte in Python

\title Programmiersprachen

Go

Python:
	objektorientierte
	dynamisch getypte
	interpretierte und interaktive
	höhere Programmiersprache

\title Interpreter vs Compiler-Sprachen

Source Code --> Interpreter --> Output    // Interpreter ist meistens ein Programm. Bsp: wie auch bei PHP oder so.

Source Code --> Compiler --> Object Code --> Executor --> Output // Executor ist gemeint, so viel wie der Computer selbst auf dem es ausgeführt wird.

Der Python-Interpreter kann auf folgende Arten gestartet werden:
	Im interaktiven Modus (ohne Angaben von Programm-Parametern)
	-> \bold{Ausdrücke} und \bold{Anweisungen} können interaktiv eingegben werden, der Interpreter wertet dies aus und druckt ggf. das Ergebnis.
	Im Skript Modus /unter Angabe einer Skript/Programm-Datei)
	-> Ein Programm (auch Skript genannt) wird eingelesen und dann ausgeführt.

\title Interaktvies Nutzen der Shell

Python Interpreter
>>>

dran denken, dass die Ausgabe in dieser Python Shell anders ist als wenn man es nur normal im Programm nutzt.
Beispiel: print(4*7) ist unterschiedlich zu 4*7 

\title Rechnen
Python kennt drei verschiedene Datentypen für Zahlen:
	int für ganze Zahlen
	float für Gleitkommazahlen
	complex für komplexe Gleitkommazahlen



Syntax
Die Schreibweise von Konstanten ist ein Aspekt der Syntay einer Programmiersprache. Sie beschreibt, welche Zeichen erlaubt sind, welche WOrte vordefiniert sind.

Operanden werden zum höheren Konvertiert. Also theoretisch

int -> float -> complexe



