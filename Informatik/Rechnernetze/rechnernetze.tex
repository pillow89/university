\title Rechnernetze
1. Organisation, Motiviation, Literatur und Schichtenmodelle
-- Christian Schindelhauer
http://cone.informatik.uni-freiburg.de/lehre/aktuell/rechnernetze-ws22


\title Übungsgruppenaufteilung
\itemize
	\item nicht durch das Campus-System
	\item neue Zeiten und Räume
	\item Sprachwahl (englisch/deutsch)
	\item Online, Präsenz, Hybrid
	\item \bold{aber: Eintrag im Campus-System notwendig}

\title Für die Neueinteilung
	\itemize
		\item nehmen Sie an der Umfrage auf ILIAS teil
			\itemize
				\item https://ilias.uni-freiburg.de/goto.php?target=svy_2868667&client_id=unifreiburg
		\itemm bis zum 20.10.2022, 16 Uhr
\title Verteilung auf die Übungsgruppen
	\itemize
		\item in der ersten Vorlesungswoche
\title Einteilung Abgabegruppen
	\itemize
		\item dann über ILIAS https://ilias.uni-freiburg.de/goto.php?target=svy_2868667&client_id=unifreiburg
		\itemm bis zum 20.10.2022, 16 Uhr
\title Verteilung auf die Übungsgruppen
	\itemize
		\item in der ersten Vorlesungswoche
\title Einteilung Abgabegruppen
	\itemize
		\item dann über ILIAS

Erstes Übungsblatt erscheint am Mittwoch 19.10.2022
- Dann immer mittwochs
- Abgabe Dienstag 23:59 Uhr in der Folgewoche.
				
Gruppenarbeiten werden erwünscht. Und gleich am Anfang auch schon alle Namen und Matrikelnummer hinzuschreiben.

*****

*****

\title Literatur

Das Buch Nr.1 zur Vorlesung:
- Computer Networking - A Top-Down Approach

Das Buch Nr.2 zur Vorlesung:
- Computer Networks, Andrew S. Tanenbaum

Auf Englisch sollte besser sein.

Datenbanken:
	Datenbanken: Grundlagen und XML-Technologien, Georg Lausen. Elsevier Spektrum Akdamischer Verlag, 2005




Types of Computer Networks

PAN
LAN
MAN
WAN

Interprocessor distance   |    Processors located in same      	Example
	1m		  |    Square meter			Personal area network		PAN
	10m		  |    Room				Local Area Network		LAN
	100m		  |    Building				"________________"
        1km		  |    Campus				"________________"
	10km		  |    City				Metropolitan area network	MAN
        100km             |    Country				Wide are network		WAN
        1000km		  |    Continent			"________________"	
	10000km		  |    Planet				The internet	


Was ist das Internet?
- Globales System aus verbundenen WANs und LANs
- Offen, systemunabhängige, ohne zentrale Kontrolle

Telefon wäre an sich auch ein Beispiel.


Datenraten
	- Werden in bit/s angegeben
	- oder Baud = Symbole/s
	- kbit/s = 10^3 Bit/s, etc

Speicher wird in Byte = 8 Bit angegeben
	- Größe meist

\title Protokolle

Menschliches Protokoll

- "Kann ich sie kurz was fragen?"
- "ja, gerne"
- "Wie viel Uhr haben wir?"
- "Viertel fünf!"
- "So spät! Vielen Dank! Tschüß"
- "Tschüß"

Spezifizierte Nachrichten
Spezifizierte Aktionen
	- aufgrund der Nachrichten

TCP connection (request) ---->
			 <---- TCP response

Vereinbarung für den Austauch von Informationen in verteilten Netzwerken
Ein Protokoll definiert:
	- das \bold{Format} gültiger Nachrichten (Syntax)
	- Regeln des Datenaustausches (GRammatik)

